% !TEX root = ../report.tex
\chapter{Probleme}
	\section{Echtzeitfähigkeit}
		Leider besitzt die derzeitige Ausarbeitung diverse kleinere Probleme, welche die Echtzeitfähigkeit des Systems gefährden.
		Diverse teile von Berechnungen werden noch wie in \ref{physik} beschrieben auf der CPU ausgeführt, während der Teil der Visualisierung bereits auf die GPU portiert wurde.
		Dies führt zu erheblichen Performanceproblemen, denn es muss bei jeder Physikberechnung (jeden Frame), die Partikeldaten zwischen GPU und CPU kopiert und synchronisiert werden.
	\section{Darstellung}
		Die Darstellung stellte sich um Laufe des Projektes als schwieriger heraus als vorher angedacht.
		Hierbei kann man die Probleme auf welche wir gestoßen sind grob in Hard- und Softwareprobleme unterscheiden.
		\subsection{Hardware}
			Trotz das wir einen Beamer von einem Grafiklabor der Hochschule zur Verfügung gestellt bekommen haben, bemerkten wir bereits bei ersten Tests, das ein großer Farbunterschied zwischen Beamer und
			Monitor vorhanden ist. Leider scheint das Spektrum unseres Beamers sehr begrenzt zu sein, so das wir einen Farbunterschied zwischen weiß und gelb kaum wahrnehmen können.		
		\subsection{Software}
			Durch die physikalische Gegebenheit das Kinekt und Beamer sich an unterschiedlichen Orten befinden, entsteht bei der Projektion zusätzlich zur Verzerrung auch noch das Problem der Verschiebung.
			Die Kalibrierung stellte sich somit schwieriger heraus als bisher gedacht, deshalb wurden aus zeitlichen Gründen der Fokus auf Aufgaben gesetzt um schnellstmöglich eine lauffähige Version zu erstellen.

\chapter{Ausblick}
	\begin{Spacing}{\mylinespace}
	Trotz das auf uns allerlei Probleme zukamen, entstand im Laufe eines Semesters eine Echtzeit Sandkastensimulation, die bereits grundlegende Funktionalität bietet. 
	Im Laufe des nächsten Semesters werden wir dann Aufgaben, welche in diesem Semester ein wenig vernachlässigt wurden wie z. B. die Kalibrierung nachbessern.
	Des Weiteren werden wir die bisherigen Physikberechnungen auf die GPU portieren um so hoffentlich wieder die Echtzeitfähigkeit des Systems zu erlangen.
	Auch neue Funktionalitäten sind geplant, welche notwendig sind um unser eigentliches \ref{Ziel} zu erreichen.
	
\end{Spacing}
\newpage
\clearpage
%% End Of Doc